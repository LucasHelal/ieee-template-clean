\documentclass[conference]{IEEEtran}

\usepackage{blindtext}
\usepackage{graphicx}
\graphicspath{ {imagens/} }
\usepackage[utf8]{inputenc}
\usepackage[english,brazilian]{babel}
\usepackage{caption}
\usepackage[cmex10]{amsmath}
\usepackage[flushleft]{threeparttable}
\usepackage{multirow}
\usepackage[table,xcdraw]{xcolor}
\usepackage{subfig}
\DeclareMathOperator*{\argmax}{argmax}
\usepackage{amssymb, amsmath}
\hyphenation{op-tical net-works semi-conduc-tor}
\usepackage{lipsum}


\begin{document}

\title{Template clean IEEE\\ para conferências da IEEE}



\author{\IEEEauthorblockN{Lucas Georges Helal}
\IEEEauthorblockA{Programa de Pós Graduacão em Ciência da Computação\\
Universidade Estadual de Maringá\\
Avenida Colombo, 5790 - Jardim Universitário\\
Maringá - Paraná - Brasil\\
Email:  lucasghelal@gmail.com}
\and
\IEEEauthorblockN{Homer Simpson}
\IEEEauthorblockA{Starfleet Academy\\
San Francisco, California 96678--2391\\
Telephone: (800) 555--1212\\
Fax: (888) 555--1212}}



\maketitle


\begin{abstract}

Resumo

\end{abstract}

\ \

\selectlanguage{english}
\begin{abstract}

Abstract

\end{abstract}

\IEEEpeerreviewmaketitle


\selectlanguage{brazilian}

\section{Introdução}
Introdução 01

\begin{figure}[h]
    \centering
    \includegraphics[width=50mm, scale=0.01]{homer.jpg}
    \captionsetup{justification=centering}
    \caption{Homer Pensativo.}
    \label{fig:homerS}
\end{figure}

\section{Fundamentação Teórica}
\label{sec:funTeorica}

\input{texto/_fundamentacao.tex}

\section{Método Proposto}
\label{sec:metodologia}

\input{texto/_metodologia.tex}


\section{Experimentos e Resultados}
\label{sec:ExperimentosResultados}

\input{texto/_resultados.tex}


\section{Conclusão}
\label{sec:conclusão}

\input{texto/_conclusao.tex}


\section*{Agradecimentos}


Agradecemos aos que contribuiram para deixar essa template clean:

\begin{itemize}
  \item Lucas Georges Helal
\end{itemize}



\begin{thebibliography}{1}

\input{texto/referencias.bib}

\end{thebibliography}




% that's all folks
\end{document}


\grid
